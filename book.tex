\documentclass[10pt,a4paper]{book}

\usepackage[utf8]{inputenc}
\usepackage{hyperref}
\usepackage{enumitem}


\title{The Art of War}
\author{Sun Tzu}
\date{\today}

\begin{document}

\maketitle
\thispagestyle{empty}

\begin{center}
{\small \textbf{The Project Gutenberg eBook of The Art of War}}
\end{center}


\noindent
This ebook is for the use of anyone anywhere in the United States and most other parts of the world at no cost and with almost no restrictions whatsoever. You may copy it, give it away or re-use it under the terms of the Project Gutenberg License included with this ebook or online at \url{www.gutenberg.org}. If you are not located in the United States, you will have to check the laws of the country where you are located before using this eBook.

\vspace{1em}

\textbf{Title:} The Art of War

\textbf{Author:} active 6th century B.C. Sunzi

\textbf{Translator:} Lionel Giles

\textbf{Release date:} May 1, 1994 [eBook \#132]

\textbf{Most recently updated:} October 29, 2024

\textbf{Language:} English

\textbf{Original publication:} 1910

\vspace{2em}

\begin{center}
{\footnotesize *** START OF THE PROJECT GUTENBERG EBOOK THE ART OF WAR ***}
\end{center}


\newpage

\thispagestyle{empty}
\begin{center}
{\Huge \textbf{Sun Tzŭ}} \\[2em]
{\Huge \textbf{on}} \\[2em]
{\Huge \textbf{The Art of War}} \\[2em]
\textbf{THE OLDEST MILITARY TREATISE IN THE WORLD} \\[1em]
Translated from the Chinese with Introduction and Critical Notes \\[1em]
BY \\[3em]
\textbf{LIONEL GILES, M.A.} \\[1em]
Assistant in the Department of Oriental Printed Books and MSS. in the British Museum \\[0.5em]
1910
\end{center}

\vspace{0em}

\noindent\hrulefill

\vspace{0em}

\begin{center}
\textit{To my brother} \\
\vspace{-0.3em}
Captain Valentine Giles, R.G. \\
\vspace{-0.3em}
\textit{in the hope that} \\
\vspace{-0.3em}
\textit{a work 2400 years old} \\
\vspace{-0.3em}
\textit{may yet contain lessons worth consideration} \\
\vspace{-0.3em}
\textit{by the soldier of today} \\
\vspace{-0.3em}
\textit{this translation} \\
\vspace{-0.3em}
\textit{is affectionately dedicated.}
\end{center}

\vspace{0em}

\noindent\hrulefill
\pagestyle{empty}

\newpage
\tableofcontents
\thispagestyle{empty}

\chapter*{Introduction}
\addcontentsline{toc}{chapter}{Introduction}

This project is a LaTeX-rendered version of *The Art of War* by Sun Tzu. It is used to demonstrate version control with Git and document preparation with LaTeX.

\chapter*{Preface to the Project Gutenberg Etext}
\addcontentsline{toc}{chapter}{Preface to the Project Gutenberg Etext}


When Lionel Giles began his translation of Sun Tzŭ’s \textit{Art of War}, the work was virtually unknown in Europe. Its introduction to Europe began in 1782 when a French Jesuit Father living in China, Joseph Amiot, acquired a copy of it, and translated it into French. It was not a good translation because, according to Dr. Giles, "[I]t contains a great deal that Sun Tzŭ did not write, and very little indeed of what he did." 

The first translation into English was published in 1905 in Tokyo by Capt. E. F. Calthrop, R.F.A. However, this translation is, in the words of Dr. Giles, "excessively bad." He goes further in this criticism: "It is not merely a question of downright blunders, from which none can hope to be wholly exempt. Omissions were frequent; hard passages were willfully distorted or slurred over. Such offenses are less pardonable. They would not be tolerated in any edition of a Latin or Greek classic, and a similar standard of honesty ought to be insisted upon in translations from Chinese." In 1908 a new edition of Capt. Calthrop’s translation was published in London. It was an improvement on the first—omissions filled up and numerous mistakes corrected—but new errors were created in the process. Dr. Giles, in justifying his translation, wrote: "It was not undertaken out of any inflated estimate of my own powers; but I could not help feeling that Sun Tzŭ deserved a better fate than had befallen him, and I knew that, at any rate, I could hardly fail to improve on the work of my predecessors." 

\newpage
Clearly, Dr. Giles’ work established much of the groundwork for the work of later translators who published their own editions. Of the later editions of the Art of War I have examined; two feature Giles’ edited translation and notes, the other two present the same basic information from the ancient Chinese commentators found in the Giles edition. Of these four, Giles’ 1910 edition is the most scholarly and presents the reader an incredible amount of information concerning Sun Tzŭ’s text, much more than any other translation.

The Giles’ edition of the Art of War, as stated above, was a scholarly work. Dr. Giles was a leading sinologue at the time and an assistant in the Department of Oriental Printed Books and Manuscripts in the British Museum. Apparently he wanted to produce a definitive edition, superior to anything else that existed and perhaps something that would become a standard translation. It was the best translation available for 50 years. But apparently there was not much interest in Sun Tzŭ in English-speaking countries since it took the start of the Second World War to renew interest in his work. Several people published unsatisfactory English translations of Sun Tzŭ. In 1944, Dr. Giles’ translation was edited and published in the United States in a series of military science books. But it wasn’t until 1963 that a good English translation (by Samuel B. Griffith and still in print) was published that was an equal to Giles’ translation. While this translation is more lucid than Dr. Giles’ translation, it lacks his copious notes that make his so interesting. 

Dr. Giles produced a work primarily intended for scholars of the Chinese civilization and language. It contains the Chinese text of Sun Tzŭ, the English translation, and voluminous notes along with numerous footnotes. Unfortunately, some of his notes and footnotes contain Chinese characters; some are completely Chinese. Thus, a conversion to a Latin alphabet etext was difficult. I did the conversion in complete ignorance of Chinese (except for what I learned while doing the conversion). Thus, I faced the difficult task of paraphrasing it while retaining as much of the important text as I could. Every paraphrase represents a loss; thus I did what I could to retain as much of the text as possible. Because the 1910 text contains a Chinese concordance, I was able to transliterate proper names, books, and the like at the risk of making the text more obscure. However, the text, on the whole, is quite satisfactory for the casual reader, a transformation made possible by conversion to an etext. However, I come away from this task with the feeling of loss because I know that someone with a background in Chinese can do a better job than I did; any such attempt would be welcomed. 

Bob Sutton 

\chapter*{Bibliography}
\addcontentsline{toc}{chapter}{Bibliography}
The following are the oldest Chinese treatises on war, after Sun Tzŭ. The notes on each have been drawn principally from the Ssu k’u ch’uan shu chien ming mu lu, ch. 9, fol. 22 sqq. 
\begin{enumerate}[label=\arabic*., leftmargin=*, labelsep=1em, itemsep=1em, align=left, nosep, wide=0pt]
    \item \textit{Wu Tzŭ}, in 1 \textit{chuan} or 6 chapters. By Wu Ch’i (d. 381 B.C.). A genuine work. See \textit{Shih Chi}, ch. 65.
    \item \textit{Ssu-ma Fa}, in 1 \textit{chuan} or 5 chapters. Wrongly attributed to Ssu-ma Jang-chu of the 6th century B.C. Its date, however, must be early, as the customs of the three ancient dynasties are constantly to be met within its pages. See \textit{Shih Chi}, ch. 64.
    The \textit{Ssu K’u Ch’uan Shu} (ch. 99, f. 1) remarks that the oldest three treatises on war, \textit{Sun Tzŭ}, \textit{Wu Tzŭ} and \textit{Ssu-ma Fa}, are, generally speaking, only concerned with things strictly military—the art of producing, collecting, training and drilling troops, and the correct theory with regard to measures of expediency, laying plans, transport of goods and the handling of soldiers—in strong contrast to later works, in which the science of war is usually blended with metaphysics, divination and magical arts in general.
    \item \textit{Liu T’ao}, in 6 \textit{chuan}, or 60 chapters. Attributed to Lu Wang (or Lu Shang, also known as T’ai Kung) of the 12th century B.C.\ [74] But its style does not belong to the era of the Three Dynasties. Lu Te-ming (550-625 A.D.) mentions the work, and enumerates the headings of the six sections so that the forgery cannot have been later than Sui dynasty.
    \item \textit{Wei Liao Tzŭ}, in 5 \textit{chuan}. Attributed to Wei Liao (4th cent. B.C.), who studied under the famous Kuei-ku Tzŭ. The work appears to have been originally in 31 chapters, whereas the text we possess contains only 24. Its matter is sound enough in the main, though the strategical devices differ considerably from those of the Warring States period. It is been furnished with a commentary by the well-known Sung philosopher Chang Tsai. 
    \item \textit{San Lueh}, in 3 \textit{chuan}. Attributed to Huang-shih Kung, a legendary personage who is said to have bestowed it on Chang Liang (\textit{d}. 187 B.C.) in an interview on a bridge. But here again, the style is not that of works dating from the Ch’in or Han period. The Han Emperor Kuang Wu [25-57 A.D.] apparently quotes from it in one of his proclamations; but the passage in question may have been inserted later on, in order to prove the genuineness of the work. We shall not be far out if we refer it to the Northern Sung period [420-478 A.D.], or somewhat earlier. 
    \item \textit{Li Wei Kung Wen Tui}, in 3 sections. Written in the form of a dialogue between T’ai Tsung and his great general Li Ching, it is usually ascribed to the latter. Competent authorities consider it a forgery, though the author was evidently well versed in the art of war.
    \item \textit{Li Ching Ping Fa} (not to be confounded with the foregoing) is a short treatise in 8 chapters, preserved in the T’ung Tien, but not published separately. This fact explains its omission from the \textit{Ssu K’u Ch’uan Shu}.
    \item \textit{Wu Ch’i Ching}, in 1 \textit{chuan}. Attributed to the legendary minister Feng Hou, with exegetical notes by Kung-sun Hung of the Han dynasty (\textit{d}. 121 B.C.), and said to have been eulogized by the celebrated general Ma Lung (\textit{d}. 300 A.D.). Yet the earliest mention of it is in the \textit{Sung Chih}. Although a forgery, the work is well put together.
\end{enumerate}

Considering the high popular estimation in which Chu-ko Liang has always been held, it is not surprising to find more than one work on war ascribed to his pen. Such are (1) the \textit{Shih Liu Ts’e} (1 \textit{chuan}), preserved in the \textit{Yung Lo Ta Tien}; (2) \textit{Chiang Yuan} (1 \textit{chuan}); and (3) \textit{Hsin Shu} (1 \textit{chuan}), which steals wholesale from Sun Tzŭ. None of these has the slightest claim to be considered genuine.

Most of the large Chinese encyclopedias contain extensive sections devoted to the literature of war. The following references may be found useful:---\\
\textit{T’ung Tien} (circa 800 A.D.), ch. 148-162. \\
\textit{T’ai P’ing Yu Lan} (983), ch. 270-359. \\
\textit{Wen Hsien Tung K’ao} (13th cent.), ch. 221. \\
\textit{Yu Hai} (13th cent.), ch. 140, 141. \\
\textit{San Ts’ai T’u Hui} (16th cent.). \\
\textit{Kuang Po Wu Chih} (1607), ch. 31, 32. \\
\textit{Ch’ien Ch’io Lei Shu} (1632), ch. 75. \\
\textit{Yuan Chien Lei Han} (1710), ch. 206-229. \\
\textit{Ku Chin T’u Shu Chi Ch’eng} (1726), section XXX, esp. ch. 81-90. \\
\textit{Hsu Wen Hsien T’ung K’ao} (1784), ch. 121-134. \\
\textit{Huang Ch’ao Ching Shih Wen Pien} (1826), ch. 76, 77.

The bibliographical sections of certain historical works also deserve mention:---
\noindent\textit{Ch’ien Han Shu}, ch. 30. \\
\textit{Sui Shu}, ch. 32-35. \\
\textit{Chiu T’ang Shu}, ch. 46, 47. \\
\textit{Hsin T’ang Shu}, ch. 57, 60. \\
\textit{Sung Shih}, ch. 202-209. \\
\textit{T’ung Chih} (circa 1150), ch. 68.

To these of course must be added the great Catalogue of the Imperial Library:--- \\
\indent\textit{Ssu K’u Ch’uan Shu Tsung Mu T’i Yao} (1790), ch. 99, 100.

\chapter{Chapter I. LAYING PLANS}

{\small
\begin{quote}
[Ts\'ao Kung, in defining the meaning of the Chinese for the title of this chapter, says it refers to the deliberations in the temple selected by the general for his temporary use, or as we should say, in his tent. See. \S 26.]
\end{quote}
}

\begin{enumerate}[leftmargin=*, label=\arabic*., wide=0pt]

\item Sun Tz\u said: The art of war is of vital importance to the State.

\item It is a matter of life and death, a road either to safety or to ruin. Hence it is a subject of inquiry which can on no account be neglected.

\item The art of war, then, is governed by five constant factors, to be taken into account in one\u2019s deliberations, when seeking to determine the conditions obtaining in the field.

\item These are: (1) The Moral Law; (2) Heaven; (3) Earth; (4) The Commander; (5) Method and discipline.

\end{enumerate}

{\small
\begin{quote}
[It appears from what follows that Sun Tz\u means by "Moral Law" a principle of harmony, not unlike the Tao of Lao Tz\u in its moral aspect. One might be tempted to render it by "morale," were it not considered as an attribute of the \textit{ruler} in \S 13.]
\end{quote}
}

\begin{enumerate}[leftmargin=*, label=\arabic*., wide=0pt, resume]
\item[5, 6.] \textit{The Moral Law} causes the people to be in complete accord with their ruler, so that they will follow him regardless of their lives, undismayed by any danger.
\end{enumerate}

{\small
\begin{quote}
[Tu Yu quotes Wang Tz"u as saying: "Without constant practice, the officers will be nervous and undecided when mustering for battle; without constant practice, the general will be wavering and irresolute when the crisis is at hand."]
\end{quote}
}

\begin{enumerate}[leftmargin=*, label=\arabic*., wide=0pt, resume]
\setcounter{enumi}{6}
\item \textit{Heaven} signifies night and day, cold and heat, times and seasons.
\end{enumerate}

{\small
\begin{quote}
[The commentators, I think, make an unnecessary mystery of two words here. Meng Shih refers to "the hard and the soft, waxing and waning" of Heaven. Wang Hsi, however, may be right in saying that what is meant is "the general economy of Heaven," including the five elements, the four seasons, wind and clouds, and other phenomena.]
\end{quote}
}

\begin{enumerate}[leftmargin=*, label=\arabic*., wide=0pt, resume]

\item \textit{Earth} comprises distances, great and small; danger and security; open ground and narrow passes; the chances of life and death.

\item \textit{The Commander} stands for the virtues of wisdom, sincerity, benevolence, courage and strictness.

\end{enumerate}

{\small
\begin{quote}
[The five cardinal virtues of the Chinese are (1) humanity or benevolence; (2) uprightness of mind; (3) self-respect, self-control, or "proper feeling;" (4) wisdom; (5) sincerity or good faith. Here "wisdom" and "sincerity" are put before "humanity or benevolence," and the two military virtues of "courage" and "strictness" substituted for "uprightness of mind" and "self-respect, self-control, or \textit{proper feeling}."]
\end{quote}
}

\begin{enumerate}[leftmargin=*, label=\arabic*., wide=0pt, resume]

\item \textit{By Method and discipline} are to be understood the marshalling of the army in its proper subdivisions, the gradations of rank among the officers, the maintenance of roads by which supplies may reach the army, and the control of military expenditure.

\item These five heads should be familiar to every general: he who knows them will be victorious; he who knows them not will fail.

\item Therefore, in your deliberations, when seeking to determine the military conditions, let them be made the basis of a comparison, in this wise:\textemdash

\end{enumerate}

\begin{enumerate}[leftmargin=*, label=\arabic*.,wide=0pt, resume]

\item \begin{enumerate}[label=(\arabic*), leftmargin=2em, wide=0pt]

\item Which of the two sovereigns is imbued with the Moral law?

{\small
\begin{quote}
[I.e., "is in harmony with his subjects." Cf. \S 5.]
\end{quote}
}

\item Which of the two generals has most ability?

\item With whom lie the advantages derived from Heaven and Earth?

{\small
\begin{quote}
[See \S\S 7, 8]
\end{quote}
}

\item On which side is discipline most rigorously enforced?

{\small
\begin{quote}
[Tu Mu alludes to the remarkable story of Ts\'ao Ts\'ao (A.D. 155-220), who was such a strict disciplinarian that once, in accordance with his own severe regulations against injury to standing crops, he condemned himself to death for having allowed his horse to shy into a field of corn! However, in lieu of losing his head, he was persuaded to satisfy his sense of justice by cutting off his hair. Ts\'ao Ts\'ao\u2019s own comment on the present passage is characteristically curt: "when you lay down a law, see that it is not disobeyed; if it is disobeyed the offender must be put to death."]
\end{quote}
}

\item Which army is the stronger?

{\small
\begin{quote}
[Morally as well as physically. As Mei Yao-ch\'en puts it, freely rendered, "esprit de corps and \textit{big battalions}."]
\end{quote}
}

\item On which side are officers and men more highly trained?

{\small
\begin{quote}
[Tu Yu quotes Wang Tz\u as saying: "Without constant practice, the officers will be nervous and undecided when mustering for battle; without constant practice, the general will be wavering and irresolute when the crisis is at hand."]
\end{quote}
}

\item In which army is there the greater constancy both in reward and punishment?

{\small
\begin{quote}
[On which side is there the most absolute certainty that merit will be properly rewarded and misdeeds summarily punished?]
\end{quote}
}

\end{enumerate}

\end{enumerate}

\begin{enumerate}[leftmargin=*, label=\arabic*.,wide=0pt, resume]

\item By means of these seven considerations I can forecast victory or defeat.

\item The general that hearkens to my counsel and acts upon it, will conquer:\textemdash let such a one be retained in command! The general that hearkens not to my counsel nor acts upon it, will suffer defeat:\textemdash let such a one be dismissed!

{\small
\begin{quote}
[The form of this paragraph reminds us that Sun Tz\u2019s treatise was composed expressly for the benefit of his patron Ho Lu, king of the Wu State.]
\end{quote}
}

\item While heeding the profit of my counsel, avail yourself also of any helpful circumstances over and beyond the ordinary rules.

\item According as circumstances are favourable, one should modify one\u2019s plans.

{\small
\begin{quote}
[Sun Tz\u, as a practical soldier, will have none of the "bookish theoric." He cautions us here not to pin our faith to abstract principles; "for," as Chang Yu puts it, "while the main laws of strategy can be stated clearly enough for the benefit of all and sundry, you must be guided by the actions of the enemy in attempting to secure a favourable position in actual warfare." On the eve of the battle of Waterloo, Lord Uxbridge, commanding the cavalry, went to the Duke of Wellington in order to learn what his plans and calculations were for the morrow, because, as he explained, he might suddenly find himself Commander-in-chief and would be unable to frame new plans in a critical moment. The Duke listened quietly and then said: "Who will attack the first tomorrow\textemdash I or Bonaparte?" "Bonaparte," replied Lord Uxbridge. "Well," continued the Duke, "Bonaparte has not given me any idea of his projects; and as my plans will depend upon his, how can you expect me to tell you what mine are?" [1] ]
\end{quote}
}

\item All warfare is based on deception.

{\small
\begin{quote}
[The truth of this pithy and profound saying will be admitted by every soldier. Col. Henderson tells us that Wellington, great in so many military qualities, was especially distinguished by "the extraordinary skill with which he concealed his movements and deceived both friend and foe."]
\end{quote}
}

\item Hence, when able to attack, we must seem unable; when using our forces, we must seem inactive; when we are near, we must make the enemy believe we are far away; when far away, we must make him believe we are near.

\item Hold out baits to entice the enemy. Feign disorder, and crush him.

{\small
\begin{quote}
[All commentators, except Chang Yu, say, "When he is in disorder, crush him." It is more natural to suppose that Sun Tz\u is still illustrating the uses of deception in war.]
\end{quote}
}

\item If he is secure at all points, be prepared for him. If he is in superior strength, evade him.

\item If your opponent is of choleric temper, seek to irritate him. Pretend to be weak, that he may grow arrogant.

{\small
\begin{quote}
[Wang Tz\u, quoted by Tu Yu, says that the good tactician plays with his adversary as a cat plays with a mouse, first feigning weakness and immobility, and then suddenly pouncing upon him.]
\end{quote}
}

\item If he is taking his ease, give him no rest.

{\small
\begin{quote}
[This is probably the meaning though Mei Yao-ch\'en has the note: "while we are taking our ease, wait for the enemy to tire himself out." The Yu Lan has "Lure him on and tire him out."]
\end{quote}
}

If his forces are united, separate them.

{\small
\begin{quote}
[Less plausible is the interpretation favoured by most of the commentators: "If sovereign and subject are in accord, put division between them."]
\end{quote}
}

\item Attack him where he is unprepared, appear where you are not expected.

\item These military devices, leading to victory, must not be divulged beforehand.

\item Now the general who wins a battle makes many calculations in his temple ere the battle is fought.

{\small
\begin{quote}
[Chang Yu tells us that in ancient times it was customary for a temple to be set apart for the use of a general who was about to take the field, in order that he might there elaborate his plan of campaign.]
\end{quote}
}

The general who loses a battle makes but few calculations beforehand. Thus do many calculations lead to victory, and few calculations to defeat: how much more no calculation at all! It is by attention to this point that I can foresee who is likely to win or lose.

{\small
\begin{quote}
[1] "Words on Wellington," by Sir. W. Fraser.
\end{quote}
}

\end{enumerate}

\chapter{Chapter II. WAGING WAR}

\begin{quote}
[Ts'ao Kung has the note: "He who wishes to fight must first count the cost," which prepares us for the discovery that the subject of the chapter is not what we might expect from the title, but is primarily a consideration of ways and means.]
\end{quote}

\begin{enumerate}[leftmargin=*, label=\arabic*., start=1, wide=0pt]
\item Sun Tz"u said: In the operations of war, where there are in the field a thousand swift chariots, as many heavy chariots, and a hundred thousand mail-clad soldiers,
\end{enumerate}

{\small
\begin{quote}
[The "swift chariots" were lightly built and, according to Chang Yu, used for the attack; the "heavy chariots" were heavier, and designed for purposes of defence. Li Ch'uan, it is true, says that the latter were light, but this seems hardly probable. It is interesting to note the analogies between early Chinese warfare and that of the Homeric Greeks. In each case, the war-chariot was the important factor, forming as it did the nucleus round which was grouped a certain number of foot-soldiers. With regard to the numbers given here, we are informed that each swift chariot was accompanied by 75 footmen, and each heavy chariot by 25 footmen, so that the whole army would be divided up into a thousand battalions, each consisting of two chariots and a hundred men.]
\end{quote}
}

with provisions enough to carry them a thousand \textit{li},

{\small
\begin{quote}
[2.78 modern \textit{li} go to a mile. The length may have varied slightly since Sun Tz"u's time.]
\end{quote}
}
The expenditure at home and at the front, including entertainment of guests, small items such as glue and paint, and sums spent on chariots and armour, will reach the total of a thousand ounces of silver per day. Such is the cost of raising an army of 100,000 men.

\begin{enumerate}[leftmargin=*, label=\arabic*.,wide=0pt, resume]
\item When you engage in actual fighting, if victory is long in coming, the men's weapons will grow dull and their ardour will be damped. If you lay siege to a town, you will exhaust your strength.
\item Again, if the campaign is protracted, the resources of the State will not be equal to the strain.
\item Now, when your weapons are dulled, your ardour damped, your strength exhausted and your treasure spent, other chieftains will spring up to take advantage of your extremity. Then no man, however wise, will be able to avert the consequences that must ensue.
\item Thus, though we have heard of stupid haste in war, cleverness has never been seen associated with long delays.
\end{enumerate}

{\small
\begin{quote}
[This concise and difficult sentence is not well explained by any of the commentators. Ts'ao Kung, Li Ch'uan, Meng Shih, Tu Yu, Tu Mu and Mei Yao-ch'en have notes to the effect that a general, though naturally stupid, may nevertheless conquer through sheer force of rapidity. Ho Shih says: "Haste may be stupid, but at any rate it saves expenditure of energy and treasure; protracted operations may be very clever, but they bring calamity in their train." Wang Hsi evades the difficulty by remarking: "Lengthy operations mean an army growing old, wealth being expended, an empty exchequer and distress among the people; true cleverness insures against the occurrence of such calamities." Chang Yu says: "So long as victory can be attained, stupid haste is preferable to clever dilatoriness." Now Sun Tz"u says nothing whatever, except possibly by implication, about ill-considered haste being better than ingenious but lengthy operations. What he does say is something much more guarded, namely that, while speed may sometimes be injudicious, tardiness can never be anything but foolish---if only because it means impoverishment to the nation. In considering the point raised here by Sun Tz"u, the classic example of Fabius Cunctator will inevitably occur to the mind. That general deliberately measured the endurance of Rome against that of Hannibal's isolated army, because it seemed to him that the latter was more likely to suffer from a long campaign in a strange country. But it is quite a moot question whether his tactics would have proved successful in the long run. Their reversal is true, led to Cannae; but this only establishes a negative presumption in their favour.]
\end{quote}
}

\begin{enumerate}[leftmargin=*, label=\arabic*., resume]
\item There is no instance of a country having benefited from prolonged warfare.
\item It is only one who is thoroughly acquainted with the evils of war that can thoroughly understand the profitable way of carrying it on.
\end{enumerate}

{\small
\begin{quote}
[That is, with rapidity. Only one who knows the disastrous effects of a long war can realize the supreme importance of rapidity in bringing it to a close. Only two commentators seem to favour this interpretation, but it fits well into the logic of the context, whereas the rendering, "He who does not know the evils of war cannot appreciate its benefits," is distinctly pointless.]
\end{quote}
}

\begin{enumerate}[leftmargin=*, label=\arabic*., resume]
\item The skilful soldier does not raise a second levy, neither are his supply-waggons loaded more than twice.
\end{enumerate}

{\small
\begin{quote}
[Once war is declared, he will not waste precious time in waiting for reinforcements, nor will he return his army back for fresh supplies, but crosses the enemy's frontier without delay. This may seem an audacious policy to recommend, but with all great strategists, from Julius Caesar to Napoleon Bonaparte, the value of time---that is, being a little ahead of your opponent---has counted for more than either numerical superiority or the nicest calculations with regard to commissariat.]
\end{quote}
}

\begin{enumerate}[leftmargin=*, label=\arabic*., wide=0pt, start=9]
\item Bring war material with you from home, but forage on the enemy. Thus the army will have food enough for its needs.
\end{enumerate}

{\small
\begin{quote}
[The Chinese word translated here as "war material" literally means "things to be used", and is meant in the widest sense. It includes all the impedimenta of an army, apart from provisions.]
\end{quote}
}

\begin{enumerate}[leftmargin=*, label=\arabic*., wide=0pt, resume]
\item Poverty of the State exchequer causes an army to be maintained by contributions from a distance. Contributing to maintain an army at a distance causes the people to be impoverished.
\end{enumerate}

{\small
\begin{quote}
[The beginning of this sentence does not balance properly with the next, though obviously intended to do so. The arrangement, moreover, is so awkward that I cannot help suspecting some corruption in the text. It never seems to occur to Chinese commentators that an emendation may be necessary for the sense, and we get no help from them there. The Chinese words Sun Tz"u used to indicate the cause of the people's impoverishment clearly have reference to some system by which the husbandmen sent their contributions of corn to the army direct. But why should it fall on them to maintain an army in this way, except because the State or Government is too poor to do so?]
\end{quote}
}

\begin{enumerate}[leftmargin=*, label=\arabic*., wide=0pt, resume]
\item On the other hand, the proximity of an army causes prices to go up; and high prices cause the people's substance to be drained away.
\end{enumerate}

{\small
\begin{quote}
[Wang Hsi says high prices occur before the army has left its own territory. Ts'ao Kung understands it of an army that has already crossed the frontier.]
\end{quote}
}

\begin{enumerate}[leftmargin=*, label=\arabic*., wide=0pt, resume]
\item When their substance is drained away, the peasantry will be afflicted by heavy exactions.
\item[13, 14.] With this loss of substance and exhaustion of strength, the homes of the people will be stripped bare, and three-tenths of their incomes will be dissipated;
\end{enumerate}

{\small
\begin{quote}
[Tu Mu and Wang Hsi agree that the people are not mulcted not of 3/10, but of 7/10, of their income. But this is hardly to be extracted from our text. Ho Shih has a characteristic tag: "The people being regarded as the essential part of the State, and \textit{food} as the people's heaven, is it not right that those in authority should value and be careful of both?"]
\end{quote}
}

\noindent while Government expenses for broken chariots, worn-out horses, breast-plates and helmets, bows and arrows, spears and shields, protective mantlets, draught-oxen and heavy waggons, will amount to four-tenths of its total revenue.

\begin{enumerate}[leftmargin=*, label=\arabic*., wide=0pt, resume]
\setcounter{enumi}{14}
\item Hence a wise general makes a point of foraging on the enemy. One cartload of the enemy's provisions is equivalent to twenty of one's own, and likewise a single \textit{picul} of his provender is equivalent to twenty from one's own store.
\end{enumerate}

{\small
\begin{quote}
[Because twenty cartloads will be consumed in the process of transporting one cartload to the front. A \textit{picul} is a unit of measure equal to 133.3 pounds (65.5 kilograms).]
\end{quote}
}

\begin{enumerate}[leftmargin=*, label=\arabic*., wide=0pt, resume]
\item Now in order to kill the enemy, our men must be roused to anger; that there may be advantage from defeating the enemy, they must have their rewards.
\end{enumerate}

{\small
\begin{quote}
[Tu Mu says: "Rewards are necessary in order to make the soldiers see the advantage of beating the enemy; thus, when you capture spoils from the enemy, they must be used as rewards, so that all your men may have a keen desire to fight, each on his own account."]
\end{quote}
}

\begin{enumerate}[leftmargin=*, label=\arabic*., wide=0pt, resume]
\item Therefore in chariot fighting, when ten or more chariots have been taken, those should be rewarded who took the first. Our own flags should be substituted for those of the enemy, and the chariots mingled and used in conjunction with ours. The captured soldiers should be kindly treated and kept.
\item This is called, using the conquered foe to augment one's own strength.
\item In war, then, let your great object be victory, not lengthy campaigns.
\end{enumerate}

{\small
\begin{quote}
[As Ho Shih remarks: "War is not a thing to be trifled with." Sun Tz"u here reiterates the main lesson which this chapter is intended to enforce.]
\end{quote}
}

\begin{enumerate}[leftmargin=*, label=\arabic*., wide=0pt, resume]
\item Thus it may be known that the leader of armies is the arbiter of the people's fate, the man on whom it depends whether the nation shall be in peace or in peril.
\end{enumerate}

\newpage
\phantomsection
\addcontentsline{toc}{chapter}{The Full Project Gutenberg License}

\begin{center}
{\footnotesize *** END OF THE PROJECT GUTENBERG EBOOK THE ART OF WAR ***}
\end{center}

\vspace{1em}

\noindent Updated editions will replace the previous one—the old editions will be renamed.

\vspace{1em}

\noindent Creating the works from print editions not protected by U.S. copyright law means that no one owns a United States copyright in these works, so the Foundation (and you!) can copy and distribute it in the United States without permission and without paying copyright royalties. Special rules, set forth in the General Terms of Use part of this license, apply to copying and distributing Project Gutenberg\texttrademark\ electronic works to protect the PROJECT Gutenberg\texttrademark\ concept and trademark. Project Gutenberg is a registered trademark, and may not be used if you charge for an eBook, except by following the terms of the trademark license, including paying royalties for use of the Project Gutenberg trademark. If you do not charge anything for copies of this eBook, complying with the trademark license is very easy. You may use this eBook for nearly any purpose such as creation of derivative works, reports, performances and research. Project Gutenberg eBooks may be modified and printed and given away—you may do practically ANYTHING in the United States with eBooks not protected by U.S. copyright law. Redistribution is subject to the trademark license, especially commercial redistribution.

\vspace{2em}
\begin{center}
\textbf{START: FULL LICENSE}

\textbf{THE FULL PROJECT GUTENBERG LICENSE}

\textit{PLEASE READ THIS BEFORE YOU DISTRIBUTE OR USE THIS WORK}
\end{center}

To protect the Project Gutenberg\texttrademark\ mission of promoting the free distribution of electronic works, by using or distributing this work (or any other work associated in any way with the phrase “Project Gutenberg”), you agree to comply with all the terms of the Full Project Gutenberg\texttrademark\ License available with this file or online at \url{www.gutenberg.org/license}.

\section*{Section 1. General Terms of Use and Redistributing Project Gutenberg\texttrademark\  electronic works}

\begin{enumerate}[label=1.\Alph*., wide=0pt, leftmargin=*, align=left]
  \item By reading or using any part of this Project Gutenberg\texttrademark\ electronic work, you indicate that you have read, understand, agree to and accept all the terms of this license and intellectual property (trademark/copyright) agreement. If you do not agree to abide by all the terms of this agreement, you must cease using and return or destroy all copies of Project Gutenberg\texttrademark\ electronic works in your possession. If you paid a fee for obtaining a copy of or access to a Project Gutenberg\texttrademark\ electronic work and you do not agree to be bound by the terms of this agreement, you may obtain a refund from the person or entity to whom you paid the fee as set forth in paragraph 1.E.8.

  \item “Project Gutenberg” is a registered trademark. It may only be used on or associated in any way with an electronic work by people who agree to be bound by the terms of this agreement. There are a few things that you can do with most Project Gutenberg\texttrademark\ electronic works even without complying with the full terms of this agreement. See paragraph 1.C below. There are a lot of things you can do with Project Gutenberg\texttrademark\ electronic works if you follow the terms of this agreement and help preserve free future access to Project Gutenberg\texttrademark\ electronic works. See paragraph 1.E below.

  \item The Project Gutenberg Literary Archive Foundation (“the Foundation” or PGLAF), owns a compilation copyright in the collection of Project Gutenberg\texttrademark\ electronic works. Nearly all the individual works in the collection are in the public domain in the United States. If an individual work is unprotected by copyright law in the United States and you are located in the United States, we do not claim a right to prevent you from copying, distributing, performing, displaying or creating derivative works based on the work as long as all references to Project Gutenberg are removed. Of course, we hope that you will support the Project Gutenberg\texttrademark\ mission of promoting free access to electronic works by freely sharing Project Gutenberg\texttrademark\ works in compliance with the terms of this agreement for keeping the Project Gutenberg\texttrademark\ name associated with the work. You can easily comply with the terms of this agreement by keeping this work in the same format with its attached full Project Gutenberg\texttrademark\ License when you share it without charge with others.

  \item The copyright laws of the place where you are located also govern what you can do with this work. Copyright laws in most countries are in a constant state of change. If you are outside the United States, check the laws of your country in addition to the terms of this agreement before downloading, copying, displaying, performing, distributing or creating derivative works based on this work or any other Project Gutenberg\texttrademark\ work. The Foundation makes no representations concerning the copyright status of any work in any country other than the United States.

  \item Unless you have removed all references to Project Gutenberg:
  \begin{enumerate}[label=\arabic*., wide=0pt, leftmargin=*, align=left]
    \item The following sentence, with active links to, or other immediate access to, the full Project Gutenberg\texttrademark\ License must appear prominently whenever any copy of a Project Gutenberg\texttrademark\ work (any work on which the phrase “Project Gutenberg” appears, or with which the phrase “Project Gutenberg” is associated) is accessed, displayed, performed, viewed, copied or distributed:
    \begin{quote}
    This eBook is for the use of anyone anywhere in the United States and most other parts of the world at no cost and with almost no restrictions whatsoever. You may copy it, give it away or re-use it under the terms of the Project Gutenberg License included with this eBook or online at \url{www.gutenberg.org}.
    \end{quote}
  \end{enumerate}
\end{enumerate}


\end{document}
